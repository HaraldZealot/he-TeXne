\documentclass[12pt, oneside]{article} 
\usepackage[xetex, a4paper, left=2cm, right=2cm, top=2cm,bottom=2cm]{geometry}
\usepackage[cm-default]{fontspec}
\usepackage{xunicode}
\usepackage{xltxtra}

\tolerance=1000
\emergencystretch=0.74cm 

\usepackage{polyglossia}
\setdefaultlanguage[spelling=modern]{russian}
\setotherlanguage{english} 
\defaultfontfeatures{Scale=MatchLowercase,Ligatures=TeX}  %% устанавливает поведение шрифтов по умолчанию  
\newfontfamily\cyrillicfont{Linux Libertine} 
\setromanfont[Mapping=tex-text]{Linux Libertine}
\setsansfont[Mapping=tex-text]{Linux Biolinum}
\setmonofont{DejaVu Sans Mono}
%\newfontfamily\cyrillicfont{Liberation Mono} 

\usepackage{makecell}

\usepackage{titlesec}
\newcommand{\sectionbreak}{\clearpage}

\renewcommand{\thesection}{\Alph{section}}
\newcount\wd    \wd=\textwidth \multiply\wd by 8 \divide\wd by 17
\newcount\wdf   \wdf=\textwidth \multiply\wdf by 16 \divide\wdf by 17

\author{Aliaksiej Stankievič}
\title{Групповые задачи на обходы}
\begin{document}
\section{Калькулятор командной строки}

Программа выдаёт приглашение командной строки на ввод. Пользователь может ввести выражение в инфиксной (обычной) форме. Выражение может содержать следующие операции: $+$, $-$\footnote{только бинарный}, $\ast$, $/$, \textasciicircum\footnote{возведение в степень}  с естественными приоритетами, также могут употребляться круглые скобки, для изменения порядка выполнения операций (любой глубины вложенности) и числа, которые представлены десятичной записью (например, 12 или 3.14 (с точкой)). Все действия над числами интерпретируются так, как если бы они все были вещественными. Программа вычисляет значение выражение и выводит его, и снова выдаёт приглашение командной строки на ввод. Выход из программы осуществляется по ключевому слову (например, exit). Также у программы должен быть второй режим работы: после имени программы через пробел пишется вычисляемое выражение, программа его вычисляет и завершает работу. Программа должна выдать консольную помощь\footnote{как при выполнении с соответствующей опцией, так и при ключевом слове в собственной командной строке}, и выдавать сообщения об ошибках.

\subsection*{Идея}
Идея программы заключается в приведении инфиксной формы в постфиксную (обратная польская запись), это осуществляется с помощью стека. Затем постфиксная форма вычисляется, опять же с помощью стека. Постфиксную форму можно хранить (а можно и не хранить), для хранения оказывается удобной очередь.

Дальнейшие развитие может быть осуществлено добавлением математических функций (sin, cos, exp и т.д.) и унарного минуса, а также вводом специальной переменной ans, ответа предыдущего вычисления.

\subsection*{Ввод}

Однострочное инфиксное выражение. Не содержит переводов строк, но может содержать пробелы, которые игнорируются.

\subsection*{Вывод}

Результат выражения, либо сообщение об ошибке (синтаксическая или математическая).

\subsection*{Пример}

\begin{tabular}{| p{\number\wdf sp} |}
\hline
\textdollar calc \\
>2+2\\
4\\
>help\\
type in mathematical expression, for example:\\
(1.4-18)/5\textasciicircum2\\
>exit\\
\textdollar calc 2\textasciicircum(1/2)\\
1.41421356\\
\textdollar calc \textendash\textendash{}help\\
CLI calculator\\
\\
usage: calc [EXPRESSION]\\
EXPRESSION must be \\correct mathematical expression, for example:\\
(1.4-18)/5\textasciicircum2\\
\textdollar \\
\hline
\end{tabular}

\section{``Пьяница''}

\subsection*{Идея}

\subsection*{Ввод}

\subsection*{Вывод}

\subsection*{Пример}

\begin{tabular}{| p{\number\wdf sp} |}
\hline
\\
\hline
\end{tabular}

\section{``Симплетрон'' с подпрограммами}

\subsection*{Идея}

\subsection*{Ввод}

\subsection*{Вывод}

\subsection*{Пример}

\begin{tabular}{| p{\number\wdf sp} |}
\hline
\\
\hline
\end{tabular}

\section{Лифт}

\subsection*{Идея}

\subsection*{Ввод}

\subsection*{Вывод}

\subsection*{Пример}

\begin{tabular}{| p{\number\wdf sp} |}
\hline
\\
\hline
\end{tabular}

\section{``Пьяница''}

\subsection*{Идея}

\subsection*{Ввод}

\subsection*{Вывод}

\subsection*{Пример}

\begin{tabular}{| p{\number\wdf sp} |}
\hline
\\
\hline
\end{tabular}

\end{document}