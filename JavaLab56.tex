\documentclass[12pt]{article} 
\usepackage[xetex, a4paper, left=2cm, right=2cm, top=2cm,bottom=2cm]{geometry}
\usepackage[cm-default]{fontspec}
\usepackage{xunicode}

%\tolerance=1000
%\emergencystretch=0.74cm 

\usepackage{polyglossia}
\setdefaultlanguage[spelling=modern]{russian}
\setotherlanguage{english} 
\defaultfontfeatures{Scale=MatchLowercase,Ligatures=TeX}  %% устанавливает поведение шрифтов по умолчанию  
\newfontfamily\cyrillicfont{Linux Libertine} 
\setromanfont[Mapping=tex-text]{Linux Libertine}
\setsansfont[Mapping=tex-text]{Linux Biolinum}
\setmonofont{DejaVu Sans Mono}
%\newfontfamily\cyrillicfont{Liberation Mono} 

%\usepackage{makecell}

%\usepackage{titlesec}
%\newcommand{\sectionbreak}{\clearpage}

%\renewcommand{\thesection}{\Alph{section}}
%\newcount\wd    \wd=\textwidth \multiply\wd by 8 \divide\wd by 17

\usepackage{minted}
\usemintedstyle{friendly}
\renewcommand\listingscaption{Код}
\newminted{bash}{frame=lines}
\newminted{c}{frame=leftline}

\usepackage[unicode, pdfborder={0 0 0 0}]{hyperref}

\author{Alaksiej Stankievič}
\title{Домашнее задание}

\begin{document}
\hypersetup{
pdftitle = {Java 5, 6},
pdfauthor = {Alaksiej Stankievič},
pdfsubject = {лабараторная работа}
}% End of hypersetup

Основная цель 5 лабораторной --- научиться работать с потоками (Tread), 
основная цель 6 --- научиться пользовать протоколом TCP/IP. Однако,  придумать 
задания в достаточном числе вариантов для 5 лабораторной, чтобы при этом они 
были достаточно простыми сложно. Поэтому я решил совместить 5 и 6 лабораторную 
в одну, так как сетевое решение с более чем двумя клиентами требует 
многопоточности, а придумать достаточное количество сетевых задач легко. Для 
небольшого числа людей (список которых я буду утверждать) будет позволено 
написать несложную чисто-многопоточную программу и сетевое приложение только с 
двумя взаимодействующими компонентами (оно не требует многопоточности). В любом 
случае дедлайн будет по сроку 6 лабораторной.

\section*{Лабораторная 5}
\subsection*{Простые задачи}


\begin{enumerate}
	\item Реализовать паралельную сортировку слиянием.
	\item Реализовать sleep <<сортировку>> целых чисел.
	\item Реализовать программу для копирования файлов, один поток читает 
второй пишет файл.
	\item Написать программу факторизации диапозона целых чисел. Каждый 
поддиапозон факторизуется в отдельном потоке.
	\item Реализовать очередь поддерживающую корректную работу с 
многопоточностью. С её помощью реализовать совместную работу двух потоков по 
чтнени чисел из файла, и вывода этих чиел в 3 степени.
\end{enumerate}


\subsection*{Сложные, но интересные задачи}
\begin{enumerate}
	\item Парсер веб-страниц.
В заданном каталоге хранятся сохраненные веб-страницы заданной структуры 
(например, сохраненные статьи с хабрахабра). Необходимо собрать данные из этих 
страниц в один файл-отчёт JSON, XML или PlainText в заданном формате (например, 
"title, author, date, text, tags[]"). Предполагается основной поток, реализующий 
выдачу задач на парсинг создаваемым потокам и записывающий результаты 
отработавших потоков в файл-отчёт и дочерние потоки, выполняющие парсинг 
(разбор) html-кода и возвращающие основному потоку данные-результат.

\item Гугломобили.
Двоичная матрица задает дорожную сеть (1 - дорога). Все дороги - с двухсторонним 
движением, по 2 полосы в каждом направлении. По дорогам едут машины: или 1 
клетка в секунду, или 0 (стоит). Машина едет по прямой, пока не проедет 
перекресток. После этого на следующем перекрестке машина собирается повернуть 
или поехать прямо (случайно). Поворот налево возможен только с левой полосы, 
направо - с правой. Приехав в тупик, машина разворачивается. Каждая машина знает 
всё о карте и других машинах. Необходимо разработать алгоритм управления 
автомобилем (дочерний поток), который получает данные от карты с автомобилями 
(основной поток) и обеспечивает эффективное и безопасное движение.

\item Поезда и семафоры.
Есть  одноколейная железнодорожная сеть с разъездами на станциях. На въезде и 
выезде со станций стоят семафоры, которые закрываются, если одноколейный 
участок пути занят. Реализовать программу управляющую несколькими поездами на 
железнодорожной сети (каждый поезд отдельный поток).
\end{enumerate}

\section*{Лабораторная 6}

Создать простою сетвую игру на 2 человек. Один запускает сервер второй --- 
клиент.

\section*{Совмещённая лабораторная 5 и 6}

Реализовать сетевое приложение имеющее один сервер и как минимум два клиента. 
Сервер обрабатывает каждого клиента в отдельном (или отдельных) потоке. В 
первую очередь ориентирую на сетевые консольные игры. Можно сделать 
полнофункциональный чат (не более 2 человек из группы с таким заданием).

\end{document}
