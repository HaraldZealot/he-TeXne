\documentclass[12pt]{article} 
\usepackage[xetex, a4paper, left=2cm, right=2cm, top=2cm,bottom=2cm]{geometry}
\usepackage[cm-default]{fontspec}
\usepackage{xunicode}

%\tolerance=1000
%\emergencystretch=0.74cm 

\usepackage{polyglossia}
\setdefaultlanguage[spelling=modern]{russian}
\setotherlanguage{english} 
\defaultfontfeatures{Scale=MatchLowercase,Ligatures=TeX}  %% устанавливает поведение шрифтов по умолчанию  
\newfontfamily\cyrillicfont{Linux Libertine} 
\setromanfont[Mapping=tex-text]{Linux Libertine}
\setsansfont[Mapping=tex-text]{Linux Biolinum}
\setmonofont{DejaVu Sans Mono}
%\newfontfamily\cyrillicfont{Liberation Mono} 

%\usepackage{makecell}

%\usepackage{titlesec}
%\newcommand{\sectionbreak}{\clearpage}

%\renewcommand{\thesection}{\Alph{section}}
%\newcount\wd    \wd=\textwidth \multiply\wd by 8 \divide\wd by 17

\usepackage{minted}
\usemintedstyle{friendly}
\renewcommand\listingscaption{Код}
\newminted{bash}{frame=lines}
\newminted{c}{frame=leftline}

\usepackage[unicode, pdfborder={0 0 0 0}]{hyperref}

\author{Alaksiej Stankievič}
\title{Домашнее задание}

\begin{document}
\hypersetup{
pdftitle = {Java 5, 6},
pdfauthor = {Alaksiej Stankievič},
pdfsubject = {лабараторная работа}
}% End of hypersetup

Основная цель 5 лабораторной --- научиться работать с потоками (Tread), 
основная цель 6 --- научиться пользовать протоколом TCP/IP. Однако,  придумать 
задания в достаточном числе вариантов для 5 лабораторной, чтобы при этом они 
были достаточно простыми сложно. Поэтому я решил совместить 5 и 6 лабораторную 
в одну, так как сетевое решение с более чем двумя клиентами требует 
многопоточности, а придумать достаточное количество сетевых задач легко. Для 
небольшого числа людей (список которых я буду утверждать) будет позволено 
написать несложную чисто-многопоточную программу и сетевое приложение только с 
двумя взаимодействующими компонентами (оно не требует многопоточности). В любом 
случае дедлайн будет по сроку 6 лабораторной.

\section*{Лабораторная 5}



\begin{enumerate}
	\item Реализовать паралельную сортировку слиянием.
	\item Реализовать sleep <<сортировку>> целых чисел.
	\item .
\end{enumerate}

\section*{Лабораторная 6}

Создать простою сетвую игру на 2 человек. Один запускает сервер второй --- 
клиент.

\section*{Совмещённая лабораторная 5 и 6}

Реализовать сетевое приложение имеющее один сервер и как минимум два клиента. 
Сервер обрабатывает каждого клиента в отдельном (или отдельных) потоке. В 
первую очередь ориентирую на сетевые консольные игры. Можно сделать 
полнофункциональный чат (не более 2 человек из группы с таким заданием).

\end{document}
