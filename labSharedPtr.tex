\documentclass[12pt, oneside]{article} 
\usepackage[xetex, a4paper, left=2cm, right=2cm, top=2cm,bottom=2cm]{geometry}
\usepackage[cm-default]{fontspec}
\usepackage{xunicode}
\usepackage{xltxtra}

%\tolerance=1000
%\emergencystretch=0.74cm 

\usepackage{polyglossia}
\setdefaultlanguage[spelling=modern]{russian}
\setotherlanguage{english} 
\defaultfontfeatures{Scale=MatchLowercase,Ligatures=TeX}  %% устанавливает поведение шрифтов по умолчанию  
\newfontfamily\cyrillicfont{Linux Libertine} 
\setromanfont[Mapping=tex-text]{Linux Libertine}
\setsansfont[Mapping=tex-text]{Linux Biolinum}
\setmonofont{DejaVu Sans Mono}
%\newfontfamily\cyrillicfont{Liberation Mono} 

%\usepackage{makecell}

%\usepackage{titlesec}
%\newcommand{\sectionbreak}{\clearpage}

%\renewcommand{\thesection}{\Alph{section}}
%\newcount\wd    \wd=\textwidth \multiply\wd by 8 \divide\wd by 17

\author{Aliaksiej Stankievič}
\title{Задачи на шаблоны, полиморфизм, исключения}
\begin{document}
 Во всех вариантах должна присутствовать полиморфная обработка исключений. Это означает, что у вас должно быть минимум два вами написанных исключения, которые унаследованы от общего предка. Также требованием задачи является то, что ваша иерархия исключений должна происходить от $std::exception$.
 
 Во всех вариантах должен присутствовать умный указатель, который реализован в виде шаблонного класса. Это должен быть указатель с семантикой подсчёта ссылок (shared pointer). Другие (unique pointer, weak pointer) формы умных указателей будут предпосылкой для бонификации. Контейнеры должны хранить умные указатели на ваши объекты, ни в коем случае не сами объекты или обычные указатели на них.
 
 Во всех вариантах вы должны использовать контейнеры из библиотеки STL, встроенные массивы и самописные контейнеры не допускаются ни в каком виде. Где это необходимо используйте итераторы, у вас обязательно должна присутствовать самописная функция которая будет выводить содержимое вашего контейнера на консоль или в файл, которая делает это с помощью итераторов. В остальных случаях вы должны максимально использовать существующие алгоритмы из STL $\#include <algorithm>$
 
 \textbf{Задания:}
 \begin{enumerate}
  \item Промоделировать работу системы контактов системы мгновенной передачи личных сообщений (skype, jabber). У вас должен быть некий список контактов (с группами), пользователи онлайн и забаненые пользователи. В процессе моделирования пользователи случайным образом меняют свой статус (онлайн, оффлайн), должно быть переключение между групповым и не групповым просмотром, а также редактирование списка забаненых.
  \item Промоделировать работу системы мгновенной передачи групповещательных сообщений (irc, чаты). У вас должен быть список присутствующих в чате. История чата, возможность передачи направленных сообщений, возможность поменяться ролями между пользователями. Для сообщений можно либо синтезировать абракадабру, либо брать куски текста из файла.
  \item Создать программу расчёта отсечений для набора геометрических фигур (полиморфизм). Фигуры должно быть минимум три вида (круг, треугольник, прямоугольник). Область отсечения задаёт пользователь (отсечение достаточно реализовать бинарное: рисуем фигуру, не рисуем).
  \item Создать программу расчёта аффинных трансформаций для геометрических фигур (полиморфизм). Фигуры должно быть минимум три вида (круг, треугольник, прямоугольник). У каждой фигуры должен быть метод рассчитывающий площадь и периметр. У вас должна быть функция рассчитывающая отношения площади к периметру. Исследуйте какие преобразования и как меняют это отношение.
  \item Создать программу расчёта частостей \textbf{слов} встречаемых в тексте (текст брать из файла, достаточно большого объёма). Вывести 20 наиболее частых слов, 40 наиболее редких, медиану, границу 30 и 70 перцентиля.
  \item Создать программу расчёта частостей \textbf{слогов} встречаемых в тексте (текст брать из файла, достаточно большого объёма). Вывести 20 наиболее частых слогов, 40 наиболее редких, медиану, границу 30 и 70 перцентиля.
  \item Создать программу расчёта оптимального маршрута. У вас есть набор пунктов отправления назначения (города), для которых известно расстояние до некоторых других пунктов. Также в каждом городе есть экземпляры перевозчиков со своими параметрами (тарифами, скоростью). Перевозчики трёх типов (полиморфизм): самолёт, автобус, поезд. Самолёт не может летать на расстояния меньшие 200 км, и за перелёт на расстояние меньшие 500 км, накладывается ``штраф'' --- стоимость увеличивается в 1.5 раз. Автобус может проезжать непрерывно в день не более 12 часов, остальное время он с пассажирами находится на отдыхе (который оценивается другим тарифом). Поезд просто везёт. Пользоваетель может вводить запросы на наименьшее время и наименьшую стоимость.
  \item Создать программу-помощник для заводчиков котов, которая позволяет выбирать лучшие пары для вязки. У вас имеется два контейнера, в одном из которых представлены коты, а в другом - кошки. Каждое животное имеет определенное количество наград, уровень здоровья и возраст.
  \item Создать программу автоматического перевода текстов о погоде. Для этого воспользваться словарем, в котором слову может соответствовать несколько синонимов переводов. Для каждого синонима указан набор слов, с котороыми он сочетается наилучшим образом.
  \item Создать объект, оперирующий нечеткой (fuzzy) логикой . Нечеткой логикой называется соответствие одного параметра, измеримого вещественной величиной второму параметру, выражаемому строкой. Например, определим, что возрасту от 18 до 30 лет соответствует слово "молодой``. Вам нужно написать программу, которая будет сравнивать два объекта, описваемых нечеткой логикой в строковом виде. Например, есть два человека, у одного возраст 20.5, а у второго - 23.3. Программа должна написать, что они оба молоды. Если же возраст разный, например, одному 20.5, а второму 60.0, то вывести, что первый моложе.
  \item Составить программу-раскраску. Вам известны rgb соответствия разным словам, например, Красный - это [250, 0, 0], а также характеристики прилагательных (краснее, зеленее, бледнее, ярче...) и дано описание областей, например "вторая область синяя, а третья бледнее, чем вторая''.
  \item Составить программу-помощник подбора логина. Если вы не зарегистрированы в системе, то логин, который вы предлагаете, подтверждается вердиктом ОК. Если такой пользователь уже существует в системе, то система предалагает вариант в виде имени с порядковым номером. Например, пусть задан запрос ``Ваня'', ``Вася'', ``Вася'', ``Вася''. Система отвечает ОК, ОК, Вася1, Вася2.
 \end{enumerate}

\end{document}
