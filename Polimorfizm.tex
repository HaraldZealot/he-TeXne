\documentclass[12pt]{report} 
\usepackage[xetex, a4paper, left=2cm, right=2cm, top=2cm,bottom=2cm]{geometry}
\usepackage[cm-default]{fontspec}
\usepackage{xunicode}
\usepackage{xltxtra}

%\tolerance=1000
%\emergencystretch=0.74cm 

\usepackage{polyglossia}
\setdefaultlanguage[spelling=modern]{russian}
\setotherlanguage{english} 
\defaultfontfeatures{Scale=MatchLowercase,Ligatures=TeX}  %% устанавливает поведение шрифтов по умолчанию  
\newfontfamily\cyrillicfont{Linux Libertine} 
\setromanfont[Mapping=tex-text]{Linux Libertine}
\setsansfont[Mapping=tex-text]{Linux Biolinum}
\setmonofont{DejaVu Sans Mono}
%\newfontfamily\cyrillicfont{Liberation Mono} 

%\usepackage{makecell}

%\usepackage{titlesec}
%\newcommand{\sectionbreak}{\clearpage}

%\renewcommand{\thesection}{\Alph{section}}
%\newcount\wd    \wd=\textwidth \multiply\wd by 8 \divide\wd by 17

\usepackage{minted}
\usemintedstyle{friendly}

\author{Alaksiej Stankievič}
\title{Лабораторная работа на полиморфизм\\(C++, Java) 3-я редакция}
\begin{document}
 \maketitle
 
 \emph{ВАЖНО!!!} Полиморфной считается только такая работа с объектом, когда с ним работают через указатель на 
предка (С++) либо ссылку на предка или интерфейс (Java).\\
 
 Во всех нижеперечисленных задачах нужно создать иерархию полиморфных объектов, а также написать приложение 
 демонстрирующее работу с этой иерархией. В заданиях описывается в первую очередь иерархия, пример использование
 можно в некоторых пределах изменять.
 \begin{enumerate}
  \item \emph{Термы польской записи}. (1 человек). Обратной польской записью  называется запись математических выражений, в которой операнды предшествуют операторам.
  Например, обычная (инфиксная) форма записи $2~+~3$ соответствует $2~3~+$ в польской. Удобство польской записи в том,
  что нет необходимости в скобках, а также, что легко вычислить выражение используя стек.
  Вот ещё несколько примеров соответствия обычной и польской записи.
  \begin{equation}
   12~+~34~*~56\Leftrightarrow{}12~34~56~*~+
  \end{equation}
  \begin{equation}
   12~*~34~+~56\Leftrightarrow{}12~34~*~56~+
  \end{equation}
  \begin{equation}
   (12~+~34)~*~56~\Leftrightarrow{}12~34~+~56~*
  \end{equation}
  \begin{equation}
   12~*~(34~+~56)~\Leftrightarrow{}12~34~56~+~*;
  \end{equation}
  \begin{equation}
   (exp(x~\textasciicircum~2~/~2)~-~sqrt(4~*~x~+~16))~*~sin(x) \\
   \Leftrightarrow x~2~\textasciicircum~2~/~exp~4~x~*~16~+~sqrt~-~x~sin~*
  \end{equation}  
  Термом называется отделимая единица текста. Так в русском языке термы это слова и знаки препинания, а польской записи
  термами будут числа, знаки операторов и имена функций. 
  
  Написать программу которая будет вычислять значение польской записи. Запись прочитать с консоли и представить в виде списка
  (или другого контейнера) термов, которые ведут себя полиморфно.
  \item \emph{Поликлиника}. (2 человека). Есть класс пациент у него внутри содержится контейнер болезней. У пациента есть 
  методы по получению   анализов и других диагностических манипуляций, которые выдают данные в зависимости от болезней и
  случайности. Есть врачи: терапевт и узкие специалисты, которые применяют различные диагностики к пациенту. 
  Необходимо поставить диагнозы пациентам.
  \item \emph{Термы brainfuck}. (1 человек). Brainfuck это эзотерический язык программирования, в котором только 8 
  команд или термов (смотрите задачу 1). Реализовать итерпретатор этого языка, прочитав программу в контейнер полиморфных
  термов.
  \item \emph{Поход в места развлечений}. (1 человек). Существуют различные категории мест развлечений (кино, театры, 
рестораны, клубы
   галереи и т.д.). Для них определены время работы, возможность свободного ухода/прихода, цены и другие характеристики.
   Необходимо из контейнера мест развлечений отобрать несколько соответствующих задаваемым критериям.
  \item \emph{Физика}. (2 или 3 человека). Вам нужно промоделировать механическое движение тел в двухмерном пространстве. 
  Существует несколько
  видов сил (например сила тяжести, сопротивление воздуха, натяжение пружины). Существует несколько видов тел (например,
  подвижные и неподвижные, точечные и стержни), на тело воздействует некоторый набор сил. Вычисление траектории производится с
  помощью интегратора (если третий человек, то несколько различных интеграторов).
  \item \emph{Диалоги}. (2 человека). У человека есть темперамент и настроение, 
которое можно назвать  внутренним состоянием. Существует,
  набор фраз с вербальным и не вербальным посылом. Вам необходимо промоделировать диалог двух человек. 
  \item \emph{Студенты}. (1 человек). Существует много шуток про студентов различных специальностей и вот несколько человек решили
  повеселится. В комнате нет одного из человек, ему можно передавать вопросы и получать ответы, вы должны угадать кто это 
  (гуманитарий, математик, физик и т.д).
  \item \emph{Пляжный дурак}. (1 человек). Создать полиморфные карты для пляжного дурака. 6 принуждает взять одну 
карту, 7  принуждает взять две карты, король пик  принуждает взять 5 карт, туз заставляет пропустить ход. Карты можно 
класть либо того же номинала либо той же масти что последняя выложенная карта.
  \item \emph{Головоломка ``угадайка''}. (2 человека). У вас завязаны глаза и вам дают предмет. На каждом этапе вы можете
  совершать некоторые действия (вам выдаётся случайно составленное меню каждый раунд). Надо за наименьшее число
  раундов определить форму и материал предмета. 
  \item \emph{Генетика}. (1 человек). У вас есть простейшее с набором некоторых свойств (способ питания, способ движения и т.д.). 
  Эти свойства кодируются генами.
  Гены бывают рецессивные и доминантные: рецессивный ген срабатывает только тогда, когда он получен от обоих родителей.
  Каждый ген кодирует не один признак, а целую совокупность, например, ген 'A' (доминантный) кодирует анаэробный метаболизм 
  и жгутик, а ген 'a' (рецессивный)- быстрое размножение и ложноножки, ген 'B' кодирует аэробный метаболизм и спячку и т.д.
  Поскольку у нас простейшее, то у них смешанная схема наследования: гены обоих родителей делятся пополам, смешиваются, а работают
  только те гены, которые образовали пару одинаковых букв, например, 'cc' или 'аА', а неравнобуквенная пара, например, 'Ba' не работает.
  Промоделировать некоторую колонию простейших на протяжении нескольких поколений.
  \item \emph{Регулярное выражение}. (1 человек). Необходимо реализовать регулярное выражение с помощью некоторой машины, состоящей из
  полиморфных блоков. Например, блок последовательного соответствия, параллельного соответствия, повторный блок и т.д. Регулярное
  выражение является упрощением стандартных регулярных выражений, содержит только последование, альтернативы, группы и простейшие
  квантификаторы.
  \item \emph{Радио-схема}. (1 человек). Вам необходимо промоделировать схему переменного тока, состоящую из полиморфных объектов (резисторы,
  конденсаторы, индуктивности, и, по желанию, диоды и транзисторы). Можно моделировать как подачу тока неизменных характеристик 
  на схему (тогда нужно получить характеристики напряжения и силы тока на каждом элементе), либо подавать на вход радиосигнал
  и получить выходной сигнал.
  \item \emph{Извлечение производных}. (1 человек). Вам дано математическое выражение в виде дерева полиморфных 
  термов (смотрите задачу 1 и схему ниже). Необходимо получить другое математическое выражение, тоже представленное деревом термов, которое будет
  производной исходного выражения по переменной $x$.
   \begin{center}
    \includegraphics{treeformula.eps}

    \textbf{Пример дерева термов}
   \end{center}

  \item \emph{Робот с командами}. (1 человек). Имеется робот или герой RPG со своими характеристиками: усталость, число жизней,
  положение в пространстве, ориентация в пространстве и т.д. Он умеет принимать полиморфные команды. Он передает командам свое 
  внутреннее состояние, а команды его изменяют. Промоделировать его поведение.
  \item \emph{Диета}. (1 человек). У продуктов питания есть калорийность, содержание белка, углеводов и жира. Существуют полиморфные
  люди, например, мужчины и женщины, с весом, ростом, возрастом, разными целями - поддерживать вес, набрать мышечную массу, сбросить
  жир и т.д. Вы должны рассчитать необходимую диету.
  \item \emph{Лабиринт} (1 человек). Реализовать лабиринт как матрицу полиморфных клеток: свободных, препятствий, 
стрелок и удвоения хода (можно придумать ещё).
  \item \emph{Блок-схема}. (1 человек). Реализовать полиморфные блоки блок-схемы: выполнения действия, ветвления, 
начало и конец.
  \item \emph{Шахматные фигуры}. (1 человек). Для шахматных фигур реализовать 
проверку возможности хода.
 \item \emph{Синхронный и асинхронный сервер} Создать абстрактный класс Server 
с публичными методами добавления нового запроса в очередь входящих запросов, 
вывода верхнего результата запроса из очереди выполненных запросов и чисто 
виртуальным методом шага.
Создать классы-наследники AsyncServer и MultithreadServer, эмулирующие (условно) 
работу серверов с асинхронной и многопоточной обработкой запросов, в которых 
перекрыт метод шага. За один вызов этого метода асинхронный сервер выбирает 
первый запрос из очереди входящих (если запрос требует одного такта, то запрос 
"выполнен" и помещается в очередь выполненных запросов; иначе запрос добавляется 
во внутреннее поле объекта (дек/список/...), предполагается что операция таких 
запросов - неблокирующая), уменьшает количество оставшихся тактов на выполнение 
всех запросов из дека на единицу. Многопоточный сервер за шаг заполняет 
внутреннюю структуру данных фиксированной длины запросами из очереди (все 
"свободные" места в ней), после чего также уменьшает количество оставшихся 
тактов на выполнение всех запросов.
Если до "выполнения" запроса остается 0 тактов, то запрос считается выполненным 
и помещается в очередь выполненных запросов.
Если очередь входящих запросов пуста, никаких действий не производится
  \item \emph{Симуляция полёта ракеты}. (1 человек). 
Небольшие импульсные ракеты удобно представить в виде списков. На одном конце 
лежит полезная нагрузка, далее идут $k$-тая, ..., 2-я, 1-я ступени, состоящие 
из топливных баков, двигателей и сепараторов. Двигатели на жидком топливе 
расходуют топливо из непосредственно прилегающего бака. Твердотопливные 
двигатели содержат топливо внутри. Сепаратор отстыковывает все элементы ракеты, 
расположенные ниже. Каждая часть обладает собственной массой, некоторые (баки и 
ТТ-двигатели) - массой хранимого топлива. Двигатели имеют параметры: расход 
топлива ($kg/s$), эффективность ($kg\cdot{}m/s^2$). По окончанию работы каждой 
части, автоматически начинает работать следующая.
Рассчитать максимальную высоту, на которую может подняться ракета, запущенная с 
Земли, заданная списком частей с их характеристиками.
 \end{enumerate}

\end{document}
