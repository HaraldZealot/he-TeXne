\documentclass[12pt]{article} 
\usepackage[xetex, a4paper, left=2cm, right=2cm, top=2cm,bottom=2cm]{geometry}
\usepackage[cm-default]{fontspec}
\usepackage{xunicode}

%\tolerance=1000
%\emergencystretch=0.74cm 

\usepackage{polyglossia}
\setdefaultlanguage[spelling=modern]{russian}
\setotherlanguage{english} 
\defaultfontfeatures{Scale=MatchLowercase,Ligatures=TeX}  %% устанавливает поведение шрифтов по умолчанию  
\newfontfamily\cyrillicfont{Linux Libertine} 
\setromanfont[Mapping=tex-text]{Linux Libertine}
\setsansfont[Mapping=tex-text]{Linux Biolinum}
\setmonofont{DejaVu Sans Mono}
%\newfontfamily\cyrillicfont{Liberation Mono} 

%\usepackage{makecell}

%\usepackage{titlesec}
%\newcommand{\sectionbreak}{\clearpage}

%\renewcommand{\thesection}{\Alph{section}}
%\newcount\wd    \wd=\textwidth \multiply\wd by 8 \divide\wd by 17

\usepackage{minted}
\usemintedstyle{friendly}
\renewcommand\listingscaption{Код}
\newminted{bash}{frame=lines}
\newminted{c}{frame=leftline}
\newminted{cpp}{frame=leftline}

\usepackage[unicode, pdfborder={0 0 0 0}]{hyperref}

\author{Alaksiej Stankievič}
\title{лабораторная работа}

\begin{document}
\hypersetup{
pdftitle = {OOP 04 lab},
pdfauthor = {Alaksiej Stankievič},
pdfsubject = {лабораторная работа}
}% End of hypersetup

\section{Композиция. Паттерн pimpl}
Композиция --- это когда класс используют экземпляр, указатель или ссылку на другой класс, для реализации своей 
функциональности\footnote{например, в классе Car четыре экземпляра класса Wheel.}. Композиция самый слабый вид связи. 

Паттерн --- это особый часто встречающийся способ решения какой-либо проблемы\footnote{мы стараемся не употреблять 
слово шаблон, чтобы не путать с соответствующей конструкцией языка C++.}.

Паттерн pimpl\footnote{от pointer to implementation, в книге GoF называется мост.} это пат-терн служащий для сильной 
модуляризации объектных программ. Он часто используется при написании библиотек, так как сильно разделяет интерфейс и 
реализацию (нет необходимости перекомпилировать программы при смене реализации). 

Паттерн pimpl основан на композиции. Создаётся два класса: класс реализации выполняющий всю работу, он скрыт от 
пользователя и поэтому может безболезненно изменяться и класс обёртка, который содержит интерфейс для внешнего 
использования (также интерфейс для потомков если необходимо) и приватный указатель на класс реализацию (отсюда и 
название). Когда пользователь вызывает методы класса обёртки, она перепоручает всю работу классу реализации.

\begin{listing}[H]
\begin{center}
\begin{cppcode}
//файл object.h
...
class Object
{
public:
    void funToAll(char value);
    ...
protected:
    int funToChildren();
    ...
private:
    class Implementation *pimpl;
};

//файл object.cpp
void Object::funToAll(char value)
{
    pimpl->funToAll(value);
}
...
class Object::Implementation
{
public:
    void funToAll(char value);
    int funToChildren();
    ...
private:
    ...
};
\end{cppcode}
\end{center}
\caption{Пример паттерна pimpl}
\label{lst:pimpl}
\end{listing}

\section{Наследование. Патерн NVI}
Наследование --- это реализация связи <<является>> между двумя классами, которая отражена в 
синтаксисе\footnote{при желании мы можем реализовать связь является и с помощью композиции (тот же pimpl) поэтому 
требования отражения в синтаксисе существенно.}. Принцип подстановки Лисков даёт критерий правомерно ли в данной 
ситуации использовать публичное наследование: наследование корректно если нам для выполнения функциональности 
заложенной в предке не нужно выполнять никаких ухищрений если нам передали потомка, то есть мы всегда безболезненно 
можем подставить потомка вместо предка\footnote{отсюда названия принципа.}.

Паттерн NVI\footnote{non virtual interface.} также служит целям модуляризации. Он не достигает такого сильного 
разделения как паттерн pimpl, но нужно затратить меньше усилий при его реализации, также он сохраняет спецификаторы 
константности. Суть паттерна состоит в том, что публичный интерфейс не виртуальный\footnote{отсюда название}, а 
следовательно при обращении по указателю будет вызван всегда метод предка. Но сам метод выполняет свою работу вызывая 
один или несколько методов которые закрыты и виртуальны.

\begin{listing}[H]
\begin{center}
\begin{cppcode}
class P
{
public:
    int process();
private:
    virtual void phase1();
    virtual int phase2();
};

int P::process()
{
    phase1();
    int result = phase2();
    phase1();
    return result;
}

class Q: public P
{
private:
    void phase1()override; // нет требования, 
    //чтобы были замещены все виртуальные функции
};



//вот как это работает
//считаем, что пользователь сделал selection
P *p = nullptr;
p = selection ? new P() : new Q();
for(int i=0; i<n; ++i)
    cout << p->process();
\end{cppcode}
\end{center}
\caption{Паттерн NVI}
\label{lst:nvi}
\end{listing}

\section{Методы итерирования по структуре}
Для того, чтобы реализовать методы вывода в консольных классах из задания нам необходимо получить доступ ко всем 
элементам структуры в некотором порядке\footnote{сделать данные защищёнными нельзя так как это нарушения инкапсуляции, 
возвращать внутренние представления тоже нельзя, так как это фактически тоже утечка доступа. Лабораторные, в которых 
будут защищённые данные или методы доступа к ним, будут отправлены на переделку.}. Соответствующие методы перебора 
информации должны быть защищёнными (уровень доступа protected).

Этого не нарушая инкапсуляции можно добиться 3 способами:
\begin{enumerate}
 \item Передать внутрь объекта стандартную структуру для заполнения, например массив (см. листинг \ref{lst:array}).
 \item Написать функцию с внутренним состоянием (см. листинг \ref{lst:statefunction}).
 \item Написать функцию, которая получает указатель на функцию обработчик, который применяется ко всем элементам 
структуры самим классом структуры (паттерн посетитель по GoF) (см. листинг \ref{lst:visitor}).
\end{enumerate}
\begin{listing}[H]
\begin{center}
\begin{cppcode}
void fillArray(Data *array, int &size);//размер изменится
//также этой функции можно поручить выделение памяти на массив
//тогда  указатель нужно предать по ссылке

//теперь массив можно использовать для вывода.
\end{cppcode}
\end{center}
\caption{Заполнение массива данными структуры}
\label{lst:array}
\end{listing}

\begin{listing}[H]
\begin{center}
\begin{cppcode}
bool currentElement(Data &returnValue, bool firstTime=false);
//функция возвращает ложь,
//если в структуре больше нет элементов и брать элемент нельзя
//На откуп программисту оставляется, 
//чем будет returnValue первый раз.
//В моей реализации returnValue первый раз
//не получает значение.


//вот как этой функцией пользоваться:
Data data;
currentElement(data, true);
while(currentElement(data))
{
    cout << data << " ";
}
\end{cppcode}
\end{center}
\caption{Функция с состоянием}
\label{lst:statefunction}
\end{listing}

\begin{listing}[H]
\begin{center}
\begin{cppcode}
//пусть написана функция обработчик
void printer(Data data)
{
    cout << data << " ";
}

//вот прототип защищённой функции применяющей обработчик
void makeVisit(void (*visitor)(Data));

//а вот так она вызывается
makeVisit(printer);
\end{cppcode}
\end{center}
\caption{Применение обработчика}
\label{lst:visitor}
\end{listing}


\section*{Задание}
Для выполнения задания необходимо изменить архитектуру программы написанной к третьей лабораторной. У основной 
структуры данных (в зависимости от задания: стек, очередь, список, дерево, граф, петля, дерамида и т.д), нужно отделить 
консольную функциональность (если она была). И реализовать эту структуру используя паттерн pimpl. От 
интерфейса-оболочки унаследовать класс, который дополняет вашу структуру данных выводом на консоль (даже если этого не 
было). От это класса унаследовать ещё один класс, у которого вывод более красив (настолько, насколько можете придумать и 
запрограммировать, минимально он должен выдавать номера элементов). При наследовании улучшенного класса от обычного 
консольного использовать паттерн NVI.

С использованием консольного и улучшенного консольного\footnote{предоставьте пользователю возможность выбора, и в 
зависимости от этого выбора создавайте объекты разных классов, работайте с ними через указатель на простой консольный 
класс.} класса реализуйте основную задачу с выводом состояния структуры на каждом шаге.

\end{document}