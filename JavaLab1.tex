\documentclass[12pt, oneside]{article} 
\usepackage[xetex, a4paper, left=2cm, right=2cm, top=2cm,bottom=2cm]{geometry}
\usepackage[cm-default]{fontspec}
\usepackage{xunicode}
\usepackage{xltxtra}

%\tolerance=1000
%\emergencystretch=0.74cm 

\usepackage{polyglossia}
\setdefaultlanguage[spelling=modern]{russian}
\setotherlanguage{english} 
\defaultfontfeatures{Scale=MatchLowercase,Ligatures=TeX}  %% устанавливает поведение шрифтов по умолчанию  
\newfontfamily\cyrillicfont{Linux Libertine} 
\setromanfont[Mapping=tex-text]{Linux Libertine}
\setsansfont[Mapping=tex-text]{Linux Biolinum}
\setmonofont{DejaVu Sans Mono}
%\newfontfamily\cyrillicfont{Liberation Mono} 

%\usepackage{makecell}

%\usepackage{titlesec}
%\newcommand{\sectionbreak}{\clearpage}

%\renewcommand{\thesection}{\Alph{section}}
%\newcount\wd    \wd=\textwidth \multiply\wd by 8 \divide\wd by 17

\usepackage{minted}
\usemintedstyle{friendly}

\usepackage{titling}
\newcommand{\subtitle}[1]{%
  \posttitle{%
    \par\end{center}
    \begin{center}\large#1\end{center}
    \vskip0.5em}%
}


\author{Aliaksiej Stankievič}
\title{Методы программирования}
\subtitle{4 семестр, задания по 1 лабораторной работе}

\begin{document}
 \maketitle
 \begin{enumerate}
  \item Длинные целые числа. Это числа с произвольным количеством цифр (например число с 1000 цифр). 
        Необходимо реализовать арифметические операции с этими числами (включая деление и взятие остатка, 
        а также смешанную операцию деления и взятия остатка).
  \item \textit{хлв} Кватернион. Это гиперкомплексные числа, их произведение антикоммутативно. Содержат 3 мнимых компоненты 
        \begin{equation}
         i^2=j^2=k^2=-1;
         i\cdot{}j=k;
         j\cdot{}k=i;
         k\cdot{}i=j.
        \end{equation}
  \item Двоичное дерево поиска. Это двоичное дерево, в котором ключи в левом поддереве меньше ключа текущей вершины, 
        а в правом больше. Реализовать операции поиска, вставки, удаления элемента и вывода дерева на консоль.
  \item Список. Двунаправленный список с операциями вставки и удаления элементов и подсписков, 
        а также сортировки списков и слияния двух отсортированных списков.
  \item Хэш-таблица. Это структура данных в которой элемент можно найти по произвольному ключу.
  \item Конечный автомат.
  \item Пирамида. Это двоичное дерево, в котором предок больше обоих потомков.
  \item \textit{хлв} Brainfuck. Реализовать интепретатор этого языка.
  \item Дробные числа. Дробные числа в виде целая часть, числитель и знаменатель, числа должны быть длинными.
  \item Timsort. Это гибридная сортировка основанная на сортировки вставками и сортировкой слиянием, характеризующаяся 
        более быстрой сортировкой частично сортированных массивов.
  \item Планировщик заданий. Есть работники и задачи. Определенная задача может выполняться после другой задачи. 
        Надо назначить работников на задачи, при этом если работник назначен на две задачи, то
        он не может выполнять обе одновременно. Вывести кто из работников делал какую
        задачу и последовательность их выполнения + общее время на выполнение всей работы.
  \item Дерамида. Это вариант двоичного дерева поиска, но у него два ключа, по одному это двоичное дерево
        поиска, а по второму пирамида.
  \item Гибридный стек. Стек сотоящий из списка массивов фиксированной длинны.
  \item Гибридная очередь. Очередь состоящая из массивов фиксированной длинны.
  \item \textit{хлв} Множество. Множество чисел или символов, с операциями объеденения, персечения, разности и симметиричной разности, 
        выводом на экран.
  \item \textit{хлв} Матрица. Двухмерная матрица произвольной размерности с типичными математическими операциями.
  \item \textit{хлв} Трехмерный вектор. Реализовать характерные операции.
  \item \textit{хлв} Многомерный вектор. Реализовать характерные операции.
  \item \textit{хлв} Многочлен. Реализовать арифмитичесике операции, вычисление производной и первообразной, нахождение нулей многочлена.
  \item Цепная дробь. Это дробь вида:
  \begin{equation}
   a_0+\frac{1}{a_1+\frac{1}{a_2+\frac{1}{a_3+\frac{1}{a_4+...}}}}
  \end{equation}
        Определить методы сложения, вычитания, умножения, деления. Вычислить значение для заданного n, x, a[n].
  \item Виртуальная машина. Упрощеённая реализация того что происходит в процессоре (собственные машинные коды, регистры). 
        Должна загружать программу в машинных кодах из файла и выполнять её.
  \item \textit{хлв} Поле для игры в трёхмерные крестики-нолики. Кубик размером 4*4*4 клеточки, побеждает выстроившей 4 крестика и или нолика в ряд.
  \item \textit{хлв} Колода карт. Для игры в дурака. вы должны уметь назначать козырную масть, тасовать колоду, осуществлять раздачу.
  \item \textit{хлв} Рулетка. Вы должны реализовать различные формы ставок (чёт, нечет, на два, на шесть и т.д.) с разными коэфициэнтами выигрыша.
 \end{enumerate}
\end{document}
 