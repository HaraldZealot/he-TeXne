\documentclass[12pt]{article} 
\usepackage[xetex, a4paper, left=2cm, right=2cm, top=2cm,bottom=2cm]{geometry}
\usepackage[cm-default]{fontspec}
\usepackage{xunicode}

%\tolerance=1000
%\emergencystretch=0.74cm 

\usepackage{polyglossia}
\setdefaultlanguage[spelling=modern]{russian}
\setotherlanguage{english} 
\defaultfontfeatures{Scale=MatchLowercase,Ligatures=TeX}  %% устанавливает поведение шрифтов по умолчанию  
\newfontfamily\cyrillicfont{Linux Libertine} 
\setromanfont[Mapping=tex-text]{Linux Libertine}
\setsansfont[Mapping=tex-text]{Linux Biolinum}
\setmonofont{DejaVu Sans Mono}
%\newfontfamily\cyrillicfont{Liberation Mono} 

%\usepackage{makecell}

%\usepackage{titlesec}
%\newcommand{\sectionbreak}{\clearpage}

%\renewcommand{\thesection}{\Alph{section}}
%\newcount\wd    \wd=\textwidth \multiply\wd by 8 \divide\wd by 17

\usepackage{minted}
\usemintedstyle{friendly}
\renewcommand\listingscaption{Код}
\newminted{bash}{frame=lines}
\newminted{c}{frame=leftline}

\usepackage[unicode, pdfborder={0 0 0 0}]{hyperref}

\author{Alaksiej Stankievič}
\title{Домашнее задание}

\begin{document}
\hypersetup{
pdftitle = {Java 3},
pdfauthor = {Alaksiej Stankievič},
pdfsubject = {лабараторная работа}
}% End of hypersetup

\section{Задание}

Реализовать программу используя стандартные класы работы со строками и 
регулярными выражениями. Использование регулярных выражений строго обязательно.

\begin{enumerate}
	\item В каждом слове текста k-ю букву заменить заданным символом. Если k 
больше длины слова, корректировку не выполнять.
	\item В русском тексте каждую букву заменить ее номером в алфавите. При 
выводе в одной строке печатать текст с двумя пробелами между буквами, в 
следующей строке внизу под каждой буквой печатать ее номер.
	\item В тексте после буквы Р, если она не последняя в слове, ошибочно 
напечатана буква А вместо О. Внести исправления в текст.
	\item В тексте слова заданной длины заменить указанной подстрокой, длина 
которой может не совпадать с длиной слова.
	\item В тексте после k-го символа вставить заданную подстроку.
	\item После каждого слова текста, заканчивающегося заданной подстрокой, 
вставить указанное слово.
	\item В зависимости от признака (0 или 1) в каждой строке текста удалить 
указанный символ везде, где он встречается, или вставить его после   k-гo 
символа.
	\item Из небольшого текста удалить все символы, кроме пробелов, не 
являющиеся буквами. Между последовательностями подряд идущих букв оставить хотя 
бы один пробел.
	\item Из текста удалить все слова заданной длины, начинающиеся на согласную 
букву.
	\item Удалить из текста его часть, заключенную между двумя символами, 
которые вводятся (например, между скобками ‘(’ и ‘)’ или между звездочками ‘*’ 
и т.п.).
	\item В тексте найти все пары слов, из которых одно является обращением 
другого.
	\item Найти и напечатать, сколько раз повторяется в тексте каждое слово, 
которое встречается в нем.
	\item В тексте найти и напечатать символы, встречающиеся наиболее часто.
	\item Найти, каких букв, гласных или согласных, больше в каждом предложении 
текста.
	\item В стихотворении найти количество слов, начинающихся и 
заканчивающихся гласной буквой.
	\item Напечатать без повторения слова текста, у которых первая и последняя 
буква совпадают.
	\item В тексте найти и напечатать все слова максимальной и все  слова 
минимальной длины.
	\item Напечатать квитанцию об оплате за телеграмму, если стоимость одного 
слова задана.
	\item В трех словах найти одинаковые буквы, которые встречаются во всех 
словах.
	\item В тексте найти первую подстроку максимальной длины, не содержащую 
букв.
	\item Задан небольшой русский текст, состоящий не более чем из 80 символов. 
Определить все согласные буквы встречающиеся не более, чем в двух словах.
	\item В тексте нет слов, начинающихся одинаковыми буквами. Напечатать слова 
текста в таком порядке, чтобы последняя буква каждого слова совпадала с первой 
буквой последующего слова. Если все слова нельзя напечатать в таком порядке, 
найти такую цепочку, состоящую из наибольшего количества слов.
	\item Найти наибольшее количество предложений текста, в которых есть 
одинаковые слова.
	\item Найти такое слово в первом предложении, которого нет ни в одном из 
остальных предложений.
	\item Во всех вопросительных предложениях текста найти и напечатать без 
повторений слава заданной длины.
	\item В каждом предложении текста поменять местами первое слово с последним, 
не изменяя длины предложения.
	\item В предложении из n слов первое слово поставить на место второго, 
второе – на место третьего, и т.д., (n-1)-ое слово – на место n-го, n-ое слово 
поставить на место первого. В исходном и преобразованном предложениях между 
словами должны быть или один пробел, или знак препинания и один пробел.
	\item Текст шифруется по следующему правилу: из исходного текста выбирается 
1, 4, 7, 10-й и т.д. (до конца текста) символы, затем 2, 5, 8, 11-й и т.д. (до 
конца текста) символы, затем 3, 6, 9, 12-й и т.д. Зашифровать заданный текст.
	\item На основании правила кодирования, описанного в предыдущем примере, 
расшифровать заданный набор символов.
	\item Напечатать слова русского текста в алфавитном порядке по первой букве. 
	\item Слова, начинающиеся с новой буквы, печатать с красной строки.
	\item Рассортировать слова русского текста по возрастанию доли гласных букв 
(отношение количества гласных к общему количеству букв в слове).
	\item Слова английского текста, начинающиеся с гласных букв, рассортировать 
л алфавитном порядке по 1-ой согласной букве слова.
	\item Все слова английского текста рассортировать по возрастанию количества 
заданной буквы в слове. Слова с одинаковым количеством расположить в алфавитном 
порядке.
	\item Ввести текст и список слов. Для каждого слова из заданного списка 
найти, сколько раз оно встречается в тексте, и рассортировать слова по убыванию 
количества вхождений.
\end{enumerate}

\section{Задание}

Создать интерфейс собственной сериализации. Для программы из второй 
лабораторной реализовать этот интерфейс для всех собственных классов. 
Воспользоваться им для сохранения и загрузки состояния в бинарный файл.

\section{Задание}

Реализовать транзитивное классовое сравнение для программы из второй 
лабороторной, продемонстрировать его работу.

\end{document}
